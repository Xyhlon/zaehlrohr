%! TeX program = lualatex
%---------------------------ALLGEMEINE IMPORTS-------------------------------------
\documentclass[12pt,english,ngerman]{scrartcl}

\input{./protokoll_template/template.latex/input/shared_preamble.tex}

    % Kopfzeile
\ihead{SS22\\11.11.2022}
\chead{\textsc{Stark} Matthias - 12004907 \\ \textsc{Philipp} Maximilian - 11839611}
\ohead{FLAB 1 \\ Zählrohr}
    % Fußzeile

\begin{document}
%\includepdf{}
\tableofcontents
\newpage

\section{Aufgabenstellung\label{Auf}}

\begin{itemize}
    \item Messung der $\alpha$, $\beta$ und $\gamma$ Strahlung ohne und mit verschiedenen dicken Abschirmungen
    \item Aufnahme der Zählrohrcharakteristik
    \item Aufnahme der Zählstatistik
    \item Bestätigung des Abstandsgesetzes
    \item Bestimmung der Endpunktsenergie über Absorbtion
    \item Aufnahme des Energiespektrums von $\beta$ Strahlung mit Magnetspektrometer
    \item Aufnahme und Kalibrierung des $\gamma$ Spektrums
    \item Aufnahme des komplexen $\gamma$ Spektrums und seinen Zerfallsprodukten
\end{itemize}

\section{Grundlagen}\label{Grund}


\section{Versuchsanordnung}\label{sec:Versuchsanordnung}

Im Laufe des Versuchs wurden 3 verschiedene Aufbauten verwendet die im Verlauf modifiziert wurden.

\subsection{Digitalzähler}\label{aufbau_Digz}

Für den ersten Teil des Versuchs wird folgender Versuchsaufbau aus \autoref{fig:digz} realisiert.
Dabei wird das Präparat in die dafür vorgesehene Halterung geschoben, hinter der sich das Zählrohr befindet, 
welches mit dem Digitalzähler verbunden ist, wodurch ein einfaches Ablesen der Counts ermöglicht wird.
Auf der optischen Bank kann der Abstand zwischen Präparat und Zählrohr variiert und abgelesen werden.
Dabei ist zu beachten, dass die abgelesene Distanz auf der optischen Bank nicht dem tatsächlichen Abstand zwischen 
Probe und Zählrohr entspricht, da sich diese nicht direkt über den Sockel befinden. Um im späteren Verlauf des Versuchs die 
Aluminiumbleche zu befestigen, wird die entsprechende Halterung auf die optische Bank gesteckt.

\begin{figure}[H]
  \begin{center}
  \includegraphics[width=0.8\textwidth]{./figures/digz.png}
	\end{center}
	\caption{Aufbau des Digitalzähler \\ 1 $\dots$ Halterung für radioaktive Quelle\\ 2 $\dots$ Zählrohr 
    \\3 $\dots$ Halterung um später das Aluminium zu Befestigen \\ 4 $\dots$ Digitalzähler 
    \\ 5 $\dots$ optische Bank um den Abstand zu variieren}
	\label{fig:digz}
    
\end{figure}

\subsection{Magnetfeldspektrometer}\label{aufbau_Magnetfeldspektrometer}

Um $\beta$ Strahlung messbar zu machen, wird folgender Aufbau aus \autoref{fig:mag} verwendet. Dabei wird das radioaktive Präparat
in das dafür vorgesehene Loch gesteckt. Durch die Spule wird ein Magnetfelds erzeugt, wodurch die Betastrahlung aufgrund von 
Lorentzkraft abgelenkt wird, weshalb die Hallsonde auch schräg zur Quelle angeordnet ist. Dies stellt sicher, dass keine Gammastrahlung
gemessen wird. Die Stärke des Magnetfelds wird durch das Netzgerät bestimmt.


\begin{figure}[H]
    \begin{center}
    \includegraphics[width=0.8\textwidth]{./figures/mag_new.png}
      \end{center}
      \caption{Aufbau des Magnetfeldspektrometers \\ 1 $\dots$ Radioaktive Quelle\\ 2 $\dots$ Hallsonde (nicht sichtbar im Foto) 
      \\3 $\dots$ Epfänger des Geiger-Müller-Zählers \\ 4 $\dots$ Anzeige des Geiger-Müller-Zählers
      \\5 $\dots$ Spule um das Magnetfeld zu erzeugen \\ 6 $\dots$ Netzgerät für das Magnetfeld 
      (Stecker um die Polung des Magnetfelds zu Ändern)
    \\7 $\dots$ Teslameter um die Stärke des Magnetfelds zu bestimmen}
      \label{fig:mag}
      
  \end{figure}


\subsection{Szintilationszähler}\label{aufbau_szinti}

Der Aufbau des Szintilationszählers ist in folgender \autoref{fig:szinti} sichtbar. Die radioaktive Quelle wird in die, dafür
vorgesehene, Halterung ober den Szintilationszähler gesteckt. Um eine Auswertung am PC zu ermöglichen, wird ein Cassy-Lab 
als Schnittstelle verwendet.

\begin{figure}[H]
    \begin{center}
    \includegraphics[width=0.8\textwidth]{./figures/szinti.png}
      \end{center}
      \caption{Aufbau des Szintilationszählers \\ 1 $\dots$ Radioaktive Quelle\\ 2 $\dots$ Szintilationszähler
      \\3 $\dots$ Spannungsgenerator \\ 4 $\dots$ Cassy-Lab um Auswertung am PC zu ermöglichen}
      
      \label{fig:szinti}
      
  \end{figure}


\section{Geräteliste}

%Geräteliste im teachcenter

\section{Versuchsdurchführung \& Messergebnisse}\label{sec:Durchführung}

\subsection{Messung der $\alpha$, $\beta$ und $\gamma$ Strahlung ohne und mit verschiedenen dicken Abschirmungen}

Um die Abschirmung Strahlungen zu Messen, wir der Versuchsbaufbau, wie in \autoref{aufbau_Digz} beschrieben, vorgenommen. 
Die Torzeit am Digitalzähler wird dabei auf \SI{10}{s} gestellt. Als radioaktive Quelle wird NA-22 verwendet, welche, wie 
bereits beim Aufbau erklärt, in die dafür vorgesehene Halterung gesteckt wird. Der Abstand zwischen der Quelle und dem Zählrohr
wird dabei so gering gewählt, dass die dickste Abschirmungsprobe problemlos dazwischen gehalten werden kann, ohne gegen
die Probe oder das Zählrohr zu stoßen. Diese Distanz zwischen der radioaktiven Quelle und dem Zählrohr wird mit einem Lineal
vermessen und beträgt \SI{15,2}{mm}. Die unterschiedlichen Abschirmungen werden der Reihe nach in den Aufbau gehalten und die 
entsprechenden Zählraten notiert, was in folgender \autoref{tab:abschirmung} sichtbar ist. Dabei ist zu Beachten, dass die 
jeweilige Abschirmung die gesammte Torzeit im Aufbau ist und man damit nicht gegen die Probe oder das Zählrohr stößt.

%tab einfügen (von zettel)
%\label{tab:abschirmung}

\subsection{Aufnahme der Zählrohrcharakteristik}

Um die Zählrohrcharakteristik zu bestimmen wird der Aufbau aus \autoref{aufbau_Magnetfeldspektrometer} realisiert.
Als radioaktive Quelle wird erneut NA-22 in die dafür vorgesehene Halterung gesteckt. Nun wird die Betriebsspannung des
Netzgerätes so lange gesenkt, bis am Geiger-Müller-Zähler kei Geräusch hörbar ist, was anzeigt, dass keine Betastrahlung auf die 
Hallsonde gelangt.


\subsection{Aufnahme der Zählstatistik}


\subsection{Bestätigung des Abstandsgesetzes}


\subsection{Bestimmung der Endpunktsenergie über Absorbtion}


\subsection{Aufnahme des Energiespektrums von $\beta$ Strahlung mit Magnetspektrometer}


\subsection{Aufnahme und Kalibrierung des $\gamma$ Spektrums}


\subsection{Aufnahme des komplexen $\gamma$ Spektrums und seinen Zerfallsprodukten}








\section{Auswertung}\label{sec:Auswertung}

\subsection{Messung der $\alpha$, $\beta$ und $\gamma$ Strahlung ohne und mit verschiedenen dicken Abschirmungen}



\subsection{Aufnahme der Zählrohrcharakteristik}




\subsection{Aufnahme der Zählstatistik}


\subsection{Bestätigung des Abstandsgesetzes}


\subsection{Bestimmung der Endpunktsenergie über Absorbtion}


\subsection{Aufnahme des Energiespektrums von $\beta$ Strahlung mit Magnetspektrometer}


\subsection{Aufnahme und Kalibrierung des $\gamma$ Spektrums}


\subsection{Aufnahme des komplexen $\gamma$ Spektrums und seinen Zerfallsprodukten}



\newpage

\printbibliography
\listoffigures
\listoftables
\end{document}
