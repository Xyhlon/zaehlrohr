%! TeX program = lualatex
%---------------------------ALLGEMEINE IMPORTS-------------------------------------
\documentclass[12pt,english,ngerman]{scrartcl}
\input{./protokoll_template/template.latex/input/shared_preamble.tex}

% Kopfzeile
\ihead{WS22\\
	11.11.2022} \chead{\textsc{Stark} Matthias - 12004907 \\
	\textsc{Philipp} Maximilian - 11839611}
\ohead{FLAB 1 \\
	Zählrohr}
% Fußzeile
\addbibresource{zaehlrohr.bib}

\begin{document}
\includepdf{./deckblatt3.pdf}
\tableofcontents
\newpage

\section{Aufgabenstellung\label{Auf}}

\begin{itemize}
	\item Messung der \(\alpha\), \(\beta\) und \(\gamma\) Strahlung ohne und mit
	      verschieden dicken Abschirmungen
	\item Aufnahme der Zählrohrcharakteristik
	\item Aufnahme der Zählstatistik
	\item Bestätigung des Abstandsgesetzes
	\item Bestimmung der Endpunktenergie über Absorption in Aluminium
	\item Aufnahme des Energiespektrums von \(\beta\) Strahlung mit Magnetspektrometer
	\item Aufnahme und Kalibrierung des \(\gamma\) Spektrums
	\item Aufnahme des komplexen \(\gamma\) Spektrums und seinen Zerfallsprodukten
\end{itemize}

\section{Grundlagen}\label{Grund}

\subsection{Radioaktivität}
Nicht alle, in der Natur vorkommenden, Isotope sind stabil und zerfallen so mit
einer gewissen Halbwertszeit \(\tau\). Bei diesen Zerfällen kann grundsätzlich
zwischen verschiedenen Arten der Zerfällen unterschieden werden.

Bei \(\alpha\)-Zerfall wird ein Heliumkern ausgestoßen, was sich schließlich
auf die Massen und Ordnungszahl auswirkt.

Bei \(\beta\)-Zerfall muss zwischen \(\beta^+\) und \(\beta^-\) unterschieden
werden. \(\beta^-\)-Zerfall wird durch die Umwandlung eines Neutrons in ein
Proton hervorgerufen, wodurch ein Elektron und ein Elektron-Antineutrino
ausgestoßen werden, um die Erhaltungssätze nicht zu verletzen.
\(\beta^+\)-Zerfall kommt in der Natur seltener vor, funktioniert aber nach dem
gleichen Prinzip, mit dem Unterschied, dass hier die Ordnungszahl erhöht wird.

Bei \(\gamma\)-Zerfall werden keine Teilchen sondern hochfrequente Wellen
abgestrahlt. Diese kommen zustande, wenn das Isotop nach \(\alpha\)- oder
\(\beta\)-Zerfall noch in einem angeregten Zustand ist, wodurch durch die
\(\gamma\)-Strahlungs-Spektren entstehen, die signifikant für bestimmte
Elemente sind, was im Laufe des Versuchs genutzt wird.

Weil diese Zerfälle immer nach einem bestimmen Schema ablaufen, können
sogenannte Zerfallsreihen angeschrieben werden, wie beispielsweise die
Zerfallsreihe von \ch{^{226}_{88}Ra} in \autoref{fig:zerfallsreihe}. Daraus
kann abgelesen werden, welche Zerfälle vorliegen und welche Halbwertszeiten
diese haben, wie häufig die Zerfälle also auftreten.

\begin{figure}[H]
	\begin{center}
		\includegraphics[width = 0.5\textwidth]{./figures/zerfallsreihe.png}
	\end{center}
	\caption{
		Zerfallsreihe \ch{^{226}_{88}Ra}~\cite{zerfallsreihera226}
	}\label{fig:zerfallsreihe}
\end{figure}

Die Intensität \(I\) radioaktiver Strahlung folgt dabei dem Abstandsgesetz,
welches folgendermaßen definiert werden kann:
\begin{equation}
	I \propto \frac{1}{l^2}
	\label{eq:abstandsgesetz}
\end{equation}

\(l\) entspricht dabei dem Abstand zur radioaktiven Quelle.

Für die Absorption von radioaktiver Strahlung gilt das Beer-Lambertsche
Absorptionsgesetz:
\begin{equation}
	I = I_0 \exp(-\mu d)
	\label{eq:beerschesgesetzt}
\end{equation}

\(I\) beschreibt dabei die Intensität der Strahlung nach der Barriere, \(I_0\)
die Anfangsintensität, \(\mu\) den Absorptionskoeffizienten der Barriere und
\(d\) die Dicke der Barriere.

Aus den Absorptionskoeffizienten kann die Ruheenergie \(E_0\) nach folgender
Formel berechnet werden:
\begin{equation}
	\frac{\mu}{\rho} = 17.6 \ E_0^{-1.39}
	\label{eq:Endpunktsenergie}
\end{equation}
\(\rho\) beschreibt dabei die Dichte der Barriere, dessen
Absorptionskoeffizient bestimmt wurde.~\cite{kuchling}

Hier wird die für dieses Experiment verwendete Gleichung für Lorentzkraft in
natürlichen Einheiten angeführt.

\begin{equation}
	\frac{p}{[\si{\mega\electronvolt}]} =  \frac{Br}{[\si{\tesla}][\si{\meter}]}
	\label{eq:lorentzimpuls}
\end{equation}

Wobei $B$ die Magnetische Flussdichte im Magnetspektrometer, $p$ der Impuls des
Teilchens und $r$ der aufbau-spezifische, approximierter Lorentzradius ist.

Weiters wurde die Energieimpulsbeziehung in natürlichen Einheiten angeführt:

\begin{equation}
	E = m_0 c^2 \left(\sqrt{\frac{ B^{2} r^{2}}{m_0 c^2} + 1} - 1 \right)
	\label{eq:energieimpulsrelation}
\end{equation}

Wobei $E$ die Energie und $m_0 c^2$ die Ruheenergie des Elektrons
ist.\cite{kuchling}

\subsection{Zählrohr}

Im Versuch wird ein sogenanntes Geiger-Müller-Zählrohr verwendet, dessen
schematischer Aufbau in \autoref{fig:zahlrohr} ersichtlich ist. Es besteht im
wesentlichen aus einem mit Zählgas gefüllten Metallrohr, durch dessen Mitte ein
dünner Draht, der als Anode fungiert, läuft. Auf diesen Draht wird eine
Spannung angelegt. Trifft nun ein zu detektierendes Teilchen auf das Fenster
des Zählrohrs, können Atome im Zählgas angeregt werden, welches durch die
angelegte Spannung zur Anode hin beschleunigt werden. Durch Stöße im Gas wird
ein Lawineneffekt ausgelöst, der schlussendlich als Peak von der Anode
verzeichnet werden kann.

\begin{figure}[H]
	\begin{center}
		\includegraphics[width = 0.5\textwidth]{./figures/zahlrohr.png}
	\end{center}
	\caption{
		Schematischer Aufbau des Zählrohrs~\cite[]{zaehlrohrvorbereitung}
	}\label{fig:zahlrohr}
\end{figure}

Der charakteristische Verlauf der Kurve des Zählrohrs ist in
\autoref{fig:zchar} ersichtlich. Dabei sind die einzelnen Bereiche zu
unterscheiden~\cite[]{gerthsen}:

\begin{enumerate}[label = \Roman*.]
	\item Hier werden die Elektronen aufgrund der angelegten Spannung zur Anode hin
	      Beschleunigt.
	\item In diesem Bereich ist eine Sättigung erreicht. Es bewegen sich also nicht alle
	      Atome direkt zur Anode.
	\item Hier ist die Spannung so hoch, dass die Atome auf den Weg zur Anode mit anderen
	      Atomen zusammenstoßen, wodurch der Lawineneffekt ausgelöst wird.
	\item In diesem Bereich befindet sich das sogenannte Geiger-Müller-Plateau. Hier ist
	      der Betrieb quasi nicht spannungsabhängig, weshalb dies auch der gewünschte
	      Messbereich ist.
	\item Eine weitere Erhöhung in diesen Bereich erhöht zwar auch die Zählrate, zerstört
	      aber auf Dauer den Zähler, weshalb dieser Bereich zu meiden ist.
\end{enumerate}

\begin{figure}[H]
	\centering
	\includegraphics[width = 0.8\textwidth]{./figures/zcharakteristik.png}
	\caption{
		Charakteristische Kurve des Zählrohrs
		für verschiedene Arten von Strahlung für die
		Bereits erklärten Spannungsbereiche~\cite[]{zaehlrohrvorbereitung}
	}\label{fig:zchar}
\end{figure}

\subsection{Magnetspektrometer}

Die Funktionsweise eines Magnetfeldspektrometer basiert auf der Lorentzkraft.
So wird \(\beta\) - Strahlung, die im Grunde aus Elektronen besteht, in einer
Kreisbahn abgelenkt und so von der \(\gamma\) - Strahlung getrennt. Der
schematische Aufbau eines Magnetfeldspektrometers ist in
\autoref{fig:magnetspektrometer} ersichtlich. In der Abbildung ist klar die
Kreisförmige 'Flugbahn' der \(\beta\) - Strahlung vom Präparat zum Zählrohr
sichtbar.\cite[]{demex2}

\begin{figure}[H]
	\begin{center}
		\includegraphics[width = 0.8\textwidth]{./figures/fig5.png}
	\end{center}
	\caption{
		Schematischer Aufbau des
		Magnetfeldspektrometers~\cite[]{zaehlrohrvorbereitung}
	}\label{fig:magnetspektrometer}
\end{figure}

\subsection{Szintilationszähler}

Der schematische Aufbau eines Szintilationszählers ist ist in
\autoref{fig:szintilationszahler} sichtbar. Grundsätzlich besteht er aus einem
Szintilator, bei dem durch die Anregung der Strahlen Photonen ausgesendet
werden, die von der Photokathode erfasst werden. Dahinter befindet sich ein
Photomultiplier, an dessen Ende schlussendlich die vielen Elektronen von der
Anode abgegriffen werden, was das erhaltene Signal darstellt.\cite{gerthsen}

\begin{figure}[H]
	\begin{center}
		\includegraphics[width = 0.8\textwidth]{./figures/szintilationszahler.png}
	\end{center}
	\caption{
		Schematischer Aufbau und
		Strahlengang im Szintillationszähler.~\cite[]{zaehlrohrvorbereitung}
	}\label{fig:szintilationszahler}
\end{figure}

\section{Versuchsanordnung}\label{sec:Versuchsanordnung}

Im Laufe des Versuchs wurden 3 verschiedene Aufbauten verwendet die im Verlauf
modifiziert wurden.

\subsection{Digitalzähler}\label{aufbau_Digz}

Für den ersten Teil des Versuchs wird folgender Versuchsaufbau aus
\autoref{fig:digz} realisiert. Dabei wird das Präparat in die dafür vorgesehene
Halterung geschoben, hinter der sich das Zählrohr befindet, welches mit dem
Digitalzähler verbunden ist, wodurch ein einfaches Ablesen der Counts
ermöglicht wird. Auf der optischen Bank kann der Abstand zwischen Präparat und
Zählrohr variiert und abgelesen werden. Dabei ist zu beachten, dass die
abgelesene Distanz auf der optischen Bank nicht dem tatsächlichen Abstand
zwischen Probe und Zählrohr entspricht, da sich diese nicht direkt über den
Sockel befinden. Um im späteren Verlauf des Versuchs die Aluminiumbleche zu
befestigen, wird die entsprechende Halterung auf die optische Bank gesteckt.

\begin{figure}[H]
	\begin{center}
		\includegraphics[width = 0.8\textwidth]{./figures/digz.png}

	\end{center}
	\caption[Aufbau des Digitalzähler]{
		Aufbau des Digitalzähler                                    \\
		1 \(\dots\) Halterung für radioaktive Quelle                \\
		2 \(\dots\) Zählrohr                                        \\
		3 \(\dots\) Halterung um später das Aluminium zu Befestigen \\
		4 \(\dots\) Digitalzähler                                   \\
		5 \(\dots\) Optische Bank um den Abstand zu variieren
	} \label{fig:digz}

\end{figure}

\subsection{Magnetfeldspektrometer}\label{sec:aufbau_Magnetfeldspektrometer}

Um \(\beta\)-Strahlung messbar zu machen, wird folgender Aufbau aus
\autoref{fig:mag} verwendet. Dabei wird das radioaktive Präparat in das dafür
vorgesehene Loch gesteckt. Durch die Spule wird ein Magnetfelds erzeugt,
wodurch die Betastrahlung aufgrund von Lorentzkraft abgelenkt wird, weshalb die
Hallsonde auch schräg zur Quelle angeordnet ist. Dies stellt sicher, dass keine
Gammastrahlung gemessen wird. Die Stärke des Magnetfelds wird durch das
Netzgerät bestimmt.

\begin{figure}[H]
	\begin{center}
		\includegraphics[width = 0.8\textwidth]{./figures/mag_new.png}
	\end{center}
	\caption[Aufbau des Magnetfeldspektrometers]{
		Aufbau des Magnetfeldspektrometers              \\
		1 \(\dots\) Radioaktive Quelle                  \\
		2 \(\dots\) Hallsonde (nicht sichtbar im Foto)  \\
		3 \(\dots\) Empfänger des Geiger-Müller-Zählers \\
		4 \(\dots\) Anzeige des Geiger-Müller-Zählers   \\
		5 \(\dots\) Spule um das Magnetfeld zu erzeugen \\
		6 \(\dots\) Netzgerät für das Magnetfeld (Stecker um die Polung des Magnetfelds
		zu Ändern)                                      \\
		7 \(\dots\) Teslameter um die Stärke des Magnetfelds zu bestimmen
	}
	\label{fig:mag}

\end{figure}

\subsection{Szintilationszähler}\label{aufbau_szinti}

Der Aufbau des Szintilationszählers ist in folgender \autoref{fig:szinti}
sichtbar. Die radioaktive Quelle wird in die, dafür vorgesehene, Halterung ober
den Szintilationszähler gesteckt. Um eine Auswertung am PC zu ermöglichen, wird
ein Cassy-Lab als Schnittstelle verwendet.

\begin{figure}[H]
	\begin{center}
		\includegraphics[width = 0.8\textwidth]{./figures/szinti.png}
	\end{center}
	\caption[Aufbau des Szintilationszählers]{
		Aufbau des Szintilationszählers \\
		1 \(\dots\) Radioaktive Quelle  \\
		2 \(\dots\) Szintilationszähler \\
		3 \(\dots\) Spannungsgenerator  \\
		4 \(\dots\) Cassy-Lab um Auswertung am PC zu ermöglichen
	} \label{fig:szinti}

\end{figure}

\section{Geräteliste}

Die Geräteliste wurde uns freundlicherweise zur Verfügung gestellt und nur um
die verwendeten Strahlungsquellen ergänzt.~\cite{zaehlrohrvorbereitung}

\begin{table}[H]
	\caption{
		Verwendete Geräte
	}
	\begin{tblr}{colspec={QQQQ}}
		Gerätetyp              & Hersteller & Typ             & Inventar-Nr \\
		Digitalzähler          & Leybold    & 57548           & 161462      \\
		Geiger-Müller-Zählrohr & Leybold    & 5240331         &             \\
		\(\beta\) Spektrometer & Phywe      &                 &             \\
		Netzgerät Universal    & Phywe      & Set Betaspektr. &             \\
		Geiger-Müller Zähler   & Phywe      & P2523200        & 79179       \\
		Spule mit Eisenkern    & Phywe      &                 &             \\
		Teslameter             & Phywe      &                 &             \\
		Hochspannungsnetzgerät & Leybold    & 52188           & 161458      \\
		Szintilationszähler    & Leybold    & 559901          & 161460      \\
		Sensor-Cassy 2         & Leybold    &                 & 161474      \\
		VKA Box                & Leybold    & 524058          & 161465      \\
		\ch{^{22}_{11}Na}      &            &                 & AG-3518     \\
		\ch{^{90}_{38}Sr}      &            &                 & AG-3676     \\
		\ch{^{226}_{88}Ra}     &            &                 & 559435      \\
	\end{tblr}
\end{table}

\section{Versuchsdurchführung \& Messergebnisse}\label{sec:Durchfuhrung}

Da in diesen Versuchen oftmals Zählraten oder Counts gemessen werden, gibt es
für diese Werte keine Unsicherheit da sie exakt sind. Aus diesem Grund wurden
mehrere Messungen aufgezeichnet und diese Messungen als normalverteilt
Angenommen, um diesen Werten eine Unsicherheit zuordnen zu können.

\subsection{Messung der \texorpdfstring{$\alpha$}{alpha}, \texorpdfstring{$\beta$}{beta} und
	\texorpdfstring{$\gamma$}{gamma} Strahlung ohne und mit verschiedenen dicken Abschirmungen}

Um die Abschirmung der Strahlungen zu Messen, wir der Versuchsaufbau, wie in
\autoref{aufbau_Digz} beschrieben, vorgenommen. Die Torzeit am Digitalzähler
wird dabei auf \SI{10}{\second} gestellt. Als radioaktive Quelle wird
\ch{^{22}_{11}Na} verwendet, welche, wie bereits beim Aufbau erklärt, in die
dafür vorgesehene Halterung gesteckt wird. Der Abstand zwischen der Quelle und
dem Zählrohr wird dabei so gering gewählt, dass die dickste Abschirmungsprobe
problemlos dazwischen gehalten werden kann, ohne gegen die Probe oder das
Zählrohr zu stoßen. Diese Distanz zwischen der radioaktiven Quelle und dem
Zählrohr wird mit einem Lineal vermessen und beträgt \SI{15(2)}{\mm}. Die
unterschiedlichen Abschirmungen werden der Reihe nach in den Aufbau gehalten
und die entsprechenden Zählraten notiert, was in folgender
\autoref{tab:abschirmung} sichtbar ist. Dabei ist zu Beachten, dass die
jeweilige Abschirmung die gesamte Torzeit im Aufbau ist und man damit nicht
gegen die Probe oder das Zählrohr stößt.

\begin{table}[H]
	\caption[Erhaltene Zählraten bei verschiedenen Abschirmungsmaterialien]{
		Erhaltene
		Zählraten bei verschiedenen Abschirmungsmaterialien bei einer Torzeit von
		\SI{10}{\second} und einem Abstand der radioaktiven Quelle von \SI{15(2)}{\mm}.
		Zählraten sind exakt.                                                                                                              \\
		\(z_{Luft} \dots\) erhaltene Zählrate ohne Abschirmung                                                                             \\
		\(z_{\mathrm{Papier}} \dots\) erhaltene Zählrate mit einem Blatt Papier als Abschirmung                                    \\
		\(z_{\mathrm{Lineal}} \dots\) erhaltene Zählrate mit einem Lineal als Abschirmung, Dicke = \SI{2.1(0.05)}{\mm}             \\
		\(z_{\mathrm{CD}} \dots\) erhaltene Zählrate mit einer CD und zugehörigen Soulcase als Abschirmung                     \\
		\(z_{\mathrm{Alu \num{0.4}}} \dots\) erhaltene Zählrate mit mit einem Aluminiumblech als Abschirmung, Dicke = \SI{0.4(0.05)}{\mm} \\
		\(z_{\mathrm{Alu \num{0.8}}} \dots\) erhaltene Zählrate mit mit einem Aluminiumblech als Abschirmung, Dicke = \SI{0.8(0.05)}{\mm} \\
		\(z_{\mathrm{Alu \num{4}}} \dots\) erhaltene Zählrate mit mit einem Aluminiumblech als Abschirmung, Dicke = \SI{4.0(0.05)}{\mm} \\
	}
	\label{tab:abschirmung}
	\begin{center}
		\begin{tblr}{colspec={S[table-format=3.1]S[table-format=3.1]S[table-format=2.1]S[table-format=1.1]S[table-format=2.1]S[table-format=2.1]S[table-format=1.1]}}
{{{$z_{\mathrm{Luft}}$ / \si{\cps}}}} & {{{$z_{\mathrm{Papier}}$ / \si{\cps}}}} & {{{$z_{\mathrm{CD}}$ / \si{\cps}}}} & {{{$z_{\mathrm{Lineal}}$ / \si{\cps}}}} & {{{$z_{\mathrm{Alu \num{0.4}}}$ / \si{\cps}}}} & {{{$z_{\mathrm{Alu \num{0.8}}}$ / \si{\cps}}}} & {{{$z_{\mathrm{Alu \num{4}}}$ / \si{\cps}}}}\\
241.6 & 167.3 & 19.4 & 9.6 & 55.1 & 15.5 & 2.3\\
250.3 & 158.7 & 21.7 & 9.8 & 56.6 & 16.3 & 2.7\\
253.0 & 148.6 & 21.4 & 9.4 & 52.9 & 14.4 & 2.9\\
248.5 & 166.5 & 22.8 & 9.6 & 61.7 & 14.5 & 2.5\\
248.3 & 164.3 & 21.3 & 9.5 & 54.2 & 15.4 & 2.4\\
\end{tblr}

	\end{center}
\end{table}

\subsection{Aufnahme der Zählrohrcharakteristik}

Um die Zählrohrcharakteristik zu bestimmen wird der Aufbau aus
\autoref{aufbau_Digz} realisiert. Als radioaktive Quelle wird erneut
\ch{^{22}_{11}Na} in die dafür vorgesehene Halterung gesteckt. Nun wird die
Betriebsspannung des Netzgerätes so lange gesenkt, bis durch den Digitalzähler
kein Geräusch hörbar ist, was anzeigt, dass keine Strahlung auf das Zählrohr
gelangt, was bei \SI{316}{\volt} der Fall war. Nun wird die Spannung
kontinuierlich erhöht, bis ein Wert von \SI{600}{\volt} erreicht ist und die
entsprechenden Counts notiert, was in folgender \autoref{tab:zaelrohrchar}
sichtbar ist.

\begin{table}[H]
	\caption[Erhaltene Zählraten für die Zählrohrcharakteristik]{
		Erhaltene Zählraten für
		die Zählrohrcharakteristik bei einer Torzeit von \SI{10}{\second} und einem
		Abstand der radioaktiven Quelle von \SI{15(2)}{\mm}. Zählraten sind exakt.                       \\
		\(U \dots\) eingestellte Betriebsspannung in \si{\volt} mit einer Unsicherheit von \SI{2}{\volt} \\
		\(z_{i} \dots\) erhaltene Zählrate bei der entsprechenden Betriebsspannung
	}
	\label{tab:zaelrohrchar}
	\begin{center}
		\begin{tblr}{colspec={S[table-format=3.1]S[table-format=3.1]S[table-format=3.1]S[table-format=3.1]}}
{{{$U$ / \si{\volt}}}} & {{{$z_{1}$ / \si{\cps}}}} & {{{$z_{2}$ / \si{\cps}}}} & {{{$z_{3}$ / \si{\cps}}}}\\
316.0 & 0.0 & 0.3 & 0.0\\
320.0 & 6.4 & 5.6 & 7.4\\
324.0 & 152.5 & 149.5 & 150.3\\
328.0 & 178.1 & 180.5 & 188.5\\
332.0 & 187.2 & 178.2 & 187.7\\
336.0 & 187.7 & 190.3 & 189.4\\
340.0 & 191.6 & 188.7 & 189.7\\
360.0 & 192.9 & 184.7 & 190.5\\
380.0 & 191.9 & 191.6 & 186.6\\
400.0 & 201.4 & 197.1 & 191.2\\
420.0 & 196.9 & 195.0 & 186.2\\
440.0 & 194.6 & 194.5 & 193.5\\
460.0 & 199.3 & 201.3 & 196.3\\
480.0 & 186.2 & 203.3 & 197.5\\
500.0 & 197.1 & 195.2 & 193.7\\
520.0 & 193.4 & 201.4 & 195.3\\
540.0 & 197.1 & 191.5 & 201.6\\
560.0 & 188.4 & 196.7 & 198.5\\
580.0 & 201.4 & 207.0 & 199.3\\
600.0 & 195.9 & 193.8 & 199.0\\
\end{tblr}

	\end{center}
\end{table}

\subsection{Aufnahme der Zählstatistik}

Um die Zählstatistik durchzuführen wird erneut der Versuchsaufbau aus
\autoref{aufbau_Digz} verwirklicht. Auch wird erneut \ch{^{22}_{11}Na} als
radioaktive Quelle verwendet. Die Torzeit beträgt für diesen Teil des Versuchs
\SI{1}{\second}. Wegen der großen Datenmenge werden die erhaltenen Counts über
den Memory Speicher des Digitalzählers direkt auf den Computer übertragen. Ein
Teil der erhaltenen Ergebnisse ist in folgender \autoref{tab:zahlstatistik}
aufgelistet.

\begin{table}[H]
	\caption[Erhaltene Counts für die Zählstatistik]{
		Tabelle in der, der Besseren Übersicht
		halber, ein Ausschnitt der erhaltenen Counts für die Zählstatistik für eine
		Torzeit von \SI{1}{\second} aufgelistet sind. Dabei sind die Counts sind exakt. \\
		\(t \dots\) entsprechende Zeit, die den Memory Speicher übergeben wird,
		in s, dabei werden die Werte der besseren Lesbarkeit halber nur auf eine
		Nachkommastelle genau angegeben. Dennoch sei hier zu bemerken, dass die
		Unsicherheit im Nanosekunden-Bereich liegt und in der Auswertung berücksichtigt
		worden ist.                                                                     \\
		\(n \dots\) verzeichnete Anzahl an Counts
	}\label{tab:zahlstatistik}
	\begin{center}
		\begin{tblr}{colspec={S[table-format=3.1]S[table-format=3.1]}}
{{{$t$ / \si{\second}}}} & {{{$n$ / 1}}}\\
1.0 & 483.0\\
2.0 & 493.0\\
3.0 & 488.0\\
4.1 & 519.0\\
5.0 & 469.0\\
6.0 & 521.0\\
7.0 & 508.0\\
8.1 & 488.0\\
9.0 & 502.0\\
10.0 & 482.0\\
\vdots & \vdots \\
438.0 & 503.0\\
439.0 & 541.0\\
440.0 & 480.0\\
441.1 & 478.0\\
442.0 & 521.0\\
443.0 & 506.0\\
444.0 & 482.0\\
445.1 & 527.0\\
446.0 & 524.0\\
447.0 & 532.0\\
448.0 & 510.0\\
449.1 & 507.0\\
450.0 & 482.0\\
\end{tblr}

	\end{center}
\end{table}

\subsection{Bestätigung des Abstandsgesetzes}

Um das Abstandsgesetz zu Bestätigen wird erneut der Versuchsaufbau aus
\autoref{aufbau_Digz} verwendet. Um die verschiedenen Abstände zu ermöglichen,
wird die radioaktive Quelle, \ch{^{90}_{38}Sr}, vom Zählrohr entfernt und die
entsprechenden Counts bei einer Torzeit von \SI{10}{\second} in
\autoref{tab:abstandsgesetz} vermerkt. Bei der Abstandsbestimmung ist zu
beachten, dass der tatsächliche Abstand zwischen Quelle und Zählrohr vermerkt
wird und nicht jener auf der optischen Bank. Um allerdings den Abstand zu
erhöhen kann auf die Skala der optischen Bank geachtet werden, da es sich um
eine Differenzmessung handelt und so ausgeschlossen werden kann, dass sich die
entstehenden Unsicherheiten durch die Messung mittels Lineal gegenläufig
auswirken.

\begin{table}[H]
	\caption[Erhaltene Zählraten bei unterschiedlichen Abständen der Quelle]{
		Erhaltene
		Zählraten bei unterschiedlichen Abständen der Quelle bei einer Torzeit von
		\SI{10}{\second}. Dabei sind die Zählraten exakt                                                                     \\
		\(l_{\mathrm{Quelle}} \dots\) Abstand der radioaktiven Quelle in cm mit einer Unsicherheit von \SI{0.2}{\cm} \\
		\(z_{i} \dots\) erhaltene Zählrate bei entsprechendem Abstand
	} \label{tab:abstandsgesetz}
	\begin{center}
		\begin{tblr}{colspec={S[table-format=2.1(1)]|S[table-format=3.1(1)]|S[table-format=3.1(1)]|S[table-format=3.1(1)]}}
{{{$l_{\mathrm{Quelle}}$}}} & {{{$z_{1}$}}} & {{{$z_{2}$}}} & {{{$z_{3}$}}}\\
{{{\si{\cm}}}} & {{{\si{\cps}}}} & {{{\si{\cps}}}} & {{{\si{\cps}}}}\\
2.0(2) & 360.9(0) & 357.7(0) & 363.8(0)\\
3.0(2) & 196.4(0) & 185.7(0) & 185.0(0)\\
4.0(2) & 119.5(0) & 123.4(0) & 108.1(0)\\
6.0(2) & 51.7(0) & 56.7(0) & 58.8(0)\\
8.0(2) & 33.1(0) & 33.8(0) & 32.7(0)\\
10.0(2) & 21.8(0) & 22.3(0) & 22.2(0)\\
\end{tblr}

	\end{center}
\end{table}

\subsection{Bestimmung der Endpunktsenergie über Absorption in Aluminium}

Um die Endpunktsenergie zu Bestimmen, wird erneut der Versuchsaufbau aus
\autoref{aufbau_Digz} verwendet. Um die unterschiedlichen Aluminiumdicken zu
realisieren, werden verschieden Dias mit unterschiedlicher Anzahl an
Aluminiumfolien in die dafür vorgesehene Halterung geschoben. Als radioaktive
Quelle wird erneut \ch{^{22}_{11}Na}, sowie eine Torzeit von \SI{10}{\second}
verwendet. Die abgelesenen Werte sind in folgender \autoref{tab:alu}
festgehalten. Dabei wurde die Unsicherheit der Plattendicke vom Betreuer mit
\SI{1}{\percent} abgeschätzt.

\begin{table}[H]
	\caption[Erhaltene Zählraten bei $\beta$-Strahlung bei verschiedenen Dicken] {
		Erhaltene
		Zählraten bei $\beta$-Strahlung bei verschiedenen Dicken einer Aluminiumplatte
		bei einer Torzeit von \SI{10}{\second}. Dabei sind die Zählraten exakte Werte.                   \\
		\(D \dots\) Dicke der Aluminiumabschirmung in $\mu$m mit einer Unsicherheit von \SI{1}{\percent} \\
		\(z_{i} \dots\) erhaltene Zählrate bei entsprechendem Abstand
	}
	\label{tab:alu}
	\begin{center}
		\begin{tblr}{colspec={S[table-format=4.1]S[table-format=3.1]S[table-format=3.1]S[table-format=3.1]}}
{{{$D$ / \si{\micro\meter}}}} & {{{$z_{1}$ / \si{\cps}}}} & {{{$z_{2}$ / \si{\cps}}}} & {{{$z_{3}$ / \si{\cps}}}}\\
7.0 & 715.7 & 721.6 & 710.8\\
14.0 & 703.9 & 700.7 & 713.2\\
21.0 & 616.6 & 614.2 & 605.2\\
22.0 & 601.0 & 603.5 & 604.8\\
28.0 & 577.3 & 584.6 & 574.1\\
42.0 & 577.6 & 582.8 & 560.3\\
50.0 & 528.1 & 527.0 & 521.9\\
55.0 & 529.5 & 525.0 & 524.9\\
85.0 & 457.2 & 456.2 & 448.9\\
100.0 & 402.7 & 415.1 & 415.0\\
110.0 & 393.7 & 401.7 & 405.4\\
165.0 & 327.1 & 336.1 & 326.5\\
200.0 & 274.3 & 290.1 & 282.8\\
220.0 & 272.2 & 267.7 & 269.6\\
345.0 & 185.5 & 193.9 & 196.7\\
400.0 & 172.6 & 163.3 & 167.9\\
600.0 & 118.7 & 116.7 & 115.6\\
800.0 & 90.2 & 93.9 & 90.3\\
1000.0 & 63.1 & 61.6 & 68.5\\
1200.0 & 52.5 & 50.8 & 47.8\\
1400.0 & 45.5 & 45.9 & 46.8\\
1600.0 & 42.7 & 39.9 & 43.5\\
2000.0 & 30.1 & 29.3 & 30.4\\
4000.0 & 16.6 & 17.9 & 15.3\\
5600.0 & 15.8 & 16.7 & 15.4\\
\end{tblr}

	\end{center}
\end{table}

\subsection{Aufnahme des Energiespektrums von \texorpdfstring{$\beta$}{beta} Strahlung mit Magnetspektrometer}

Um das Energiespektrum der \(\beta\) Strahlung zu bestimmen wird der Aufbau aus
\autoref{sec:aufbau_Magnetfeldspektrometer} realisiert. Als radioaktive Quelle
wird erneut \ch{^{22}_{11}Na} in die dafür vorgesehene Halterung gesteckt. Nun
wird die Betriebsspannung des Netzgerätes so lange gesenkt, bis das erzeugte
Magnetfeld in etwa \SI{5}{\milli\tesla} entspricht. Bei den Anschlüssen der
Spule ist dabei zu beachten, dass das Magnetfeld richtig gepolt ist, um die
Strahlung in die richtige Richtung abzulenken. Nun wird die Spannung durch
betätigen des entsprechenden Rades kontinuierlich erhöht und die jeweiligen
Zerfälle bei einer Torzeit von \SI{100}{\second} gemeinsam mit dem jeweiligen
Wert des Magnetfelds in folgender \autoref{tab:magnetspektrometer} aufgelistet.
Dabei ist auch wichtig, dass die Hintergrundstrahlung im entsprechenden Gebäude
gemessen wird, indem die selbe Messung auch einmal ohne eingelegte radioaktive
Quelle durchgeführt wird, wodurch eine Hintergrundstrahlung von 23 Zerfällen in
der entsprechenden Torzeit vermerkt wird. Der dabei erhaltenen Wert muss dann
von den vorherigen Werten abgezogen werden.

Der Radius ist durch den Aufbau bestimmt, was Unterlagen des Geräts zu
entnehmen ist und einen Wert von \SI{5.00(3)}{\cm}
hat.\cite{zaehlrohrvorbereitung}

\begin{table}[H]
	\caption[Verzeichnete Zerfälle bei entsprechendem Magnetfeld]{
		Verzeichnete Zerfälle bei
		entsprechendem Magnetfeld bei einer Torzeit von \SI{100}{\second}. Dabei sind
		die Counts exakte Werte.                \\
		\(B \dots\) Stärke des Magnetfelds in \si{\milli\tesla} mit einer
		Unsicherheit von \SI{0.2}{\milli\tesla} \\
		\(n \dots\) erhaltene bei erhaltene Anzahl an
		Zerfällen bei entsprechendem Magnetfeld
	}\label{tab:magnetspektrometer}
	\begin{center}
		\begin{tblr}{colspec={S[table-format=2.1]S[table-format=3.3]}}
{{{$B$ / \si{\milli\tesla}}}} & {{{$n$ / 1}}}\\
4.5 & 130.000\\
10.1 & 175.000\\
14.9 & 214.000\\
20.2 & 260.000\\
24.9 & 300.000\\
30.0 & 342.000\\
35.0 & 347.000\\
40.1 & 380.000\\
45.1 & 360.000\\
50.0 & 316.000\\
55.0 & 260.000\\
60.0 & 212.000\\
65.0 & 176.000\\
\end{tblr}

	\end{center}
\end{table}

\subsection{Aufnahme und Kalibrierung des \texorpdfstring{$\gamma$}{gamma}
	Spektrums}\label{sec:aufname_kalibrierungsspektrum}

Um das \(\gamma\) Spektrum zu kalibrieren wird der Versuch wie in
\autoref{aufbau_szinti} erklärt aufgebaut. Um das Referenzspektrum aufzunehmen
wird eine \ch{^{137}_{55}Cs} Quelle in die Halterung eingesetzt. Für die
Hochspannung wird dabei ein Wert von \SI{0.73}{\kilo\volt} eingestellt.

Mithilfe des Cassy-Labs werden die erhaltenen Daten direkt an den Computer
gesendet, wodurch die entsprechenden Spektren geplottet werden können. Da hier
die Werte für die Peaks bekannt sind, kann so eine Kalibrierungskurve erzeugt
werden, was in folgender \autoref{fig:kalibrierung} sichtbar ist.

\begin{figure}[H]
	\begin{center}
		\includegraphics[width = 0.95\textwidth]{./figures/c137kalibrierung.png}
	\end{center}
	\caption[Kalibrierungsmessung einer \ch{^{137}_{55}Cs}]{
		Diese Graphik beinhaltet, das
		zur Kalibrierung verwendete Referenzspektrum einer \ch{^{137}_{55}Cs} Probe.
		Zudem wurden die charakteristischen Energiepeaks dieser Probe markiert.
	}\label{fig:kalibrierung}
\end{figure}

\subsection{Aufnahme des komplexen \texorpdfstring{$\gamma$}{gamma}
	Spektrums und seinen Zerfallsprodukten}\label{sec:aufname_Ra_zerfallsreihe}

Der Versuch wird, wie zuvor beschrieben, wie in \autoref{aufbau_szinti}
aufgebaut, auch wird erneut eine Spannung von \SI{0.73}{\kilo\volt} verwendet.
Als radioaktive Quelle wird für diesen Teil des Versuchs das zu vermessende
\ch{^{226}_{88}Ra} verwendet. Auch diese Werte werden auf den Computer
übertragen und den zuvor erzeugten Plot bei einer Laufzeit von
\SI{2400}{\second} beigefügt. Anhand des zuvor bestimmten Referenzspektrums
können nun die Peaks des \ch{^{226}_{88}Ra} Spektrums vermessen werden.

\section{Auswertung}\label{sec:Auswertung}

Um zu sehen wie sich die Unsicherheit der Messungen bis in die Ergebnisse
fortpflanzt, ist erweiterte Gauss-Methode verwendet worden. Die Grundlagen
dieser Methode stammen von den Powerpointfolien von
GUM~\cite{WolfgangKessel2004}. Für die Auswertung ist die Progammiersprache
Python im speziellen die Pakete \verb#labtool-ex2#, \verb#pandas#,
\verb#sympy#, zur Hilfe genommen worden. Um höchstmögliche Genauigkeit zu
garantieren wird erst bei der Darstellung der Wert in Tabellen gerundet.

\subsection{Messung der \texorpdfstring{$\alpha$}{alpha}, \texorpdfstring{$\beta$}{beta} und
	\texorpdfstring{$\gamma$}{gamma} Strahlung ohne und mit verschiedenen dicken Abschirmungen}

Da es sich bei diesem Teil des Versuchs um eine rein qualitative Aussage
handelt, wird keine explizite Auswertung durchgeführt und die erhaltenen
Messwerte in \autoref{sec:diskussion} analysiert.

\subsection{Aufnahme der Zählrohrcharakteristik}

Die Daten der Zählraten $z_i$ aus \autoref{tab:zaelrohrchar} werden zunächst
gemittelt und dessen Standarderror berechnet. Die durch diese Operation
erhaltenen Zählraten werden über den Spannungen aus derselben
\autoref{tab:zaelrohrchar} aufgetragen, das Resultat ist in
\autoref{fig:zaelrohrchar} ersichtlich. Zudem wird ein Linearer Fit, bei dem,
für das Zählrohr charakteristische Plateau, gemacht. Dafür wurden alle
Datenpunkte unter der \SI{160}{1} Zählratengrenze ignoriert, da diese nicht
Teil des Plateaus sind.

\begin{figure}[H]
	\begin{center}
		\includegraphics[width = 0.95\textwidth]{figures/charakteristik.pdf}
	\end{center}
	\caption[Aufnahme der Zählrohrcharakteristik bei \ch{^{22}_{11}Na} Probe mit linearem
		Fit]{
		In dieser Graphik ist die Zählrohrcharakteristik dargestellt worden,
		indem die Zählrate $z$ über die Spannung $U$ in Volt aufgetragen wurde. Die
		Daten wurden \autoref{tab:zaelrohrchar} entnommen. Weiters wurde das
		charakteristische Plateau des Zählrohr linear gefittet. Hier ist $m$ die
		Steigung der Geraden in $\si{\per\volt}$ und $b$ entspricht dem Schnitt der
		Ordinate
	}\label{fig:zaelrohrchar}
\end{figure}

\subsection{Aufnahme der Zählstatistik}

Die gesamte aufgenommene Messreihe der Zählraten aus
\autoref{tab:zahlstatistik} wurde nun in Klassen mit einer konstanten 5er und
10er Breite $h$ unterteilt und dann als Histogramme dargestellt. Das Histogramm
mit der konstanten 5er Breite ist in \autoref{fig:5statistik} ersichtlich,
jenes mit der 10er Breite ist in \autoref{fig:10statistik} zu finden.

Des Weiteren wurden mittels der Standardabweichung und dem Mittelwert der
Klassen eine Normalverteilung aufgestellt und diese über den Verteilungsraum
der Messwerte geplottet.

Damit der Vergleich zwischen der Normalverteilung und dem Histogrammen klar
ersichtlich ist, werden die Histogramme auf 1 normiert. Um auf die Absolute
Häufigkeit zu kommen muss die Relative Häufigkeit $p$ mit dem Stichprobenumfang
$N$ und der Breite der $h$ multipliziert werden. Alle relevanten Größen für
diese Umrechnung sind in den jeweiligen Bildern ersichtlich.

\begin{figure}[H]
	\begin{center}
		\includegraphics[width=0.95\textwidth]{./figures/5statistik.pdf}
	\end{center}
	\caption[Histogramm der Zählstatistik mit Klassen der Größe 5]{
		Diese Graphik beinhaltet
		das normierte Histogramm der Messreihe aus \autoref{tab:zahlstatistik}. Hier
		entspricht $N$ dem Stichprobenumfang und $h$ der Breite der Klassen des
		Histogramms. Darüber hinaus wurde eine Normalverteilung angelegt unter
		Verwendung des Mittelwert $\mu$ \& der Standardabweichung $\sigma$ vom
		Messreihensatz. Alle hier in der Box angeführten Variablen haben Einheit 1
	}\label{fig:5statistik}
\end{figure}

\begin{figure}[H]
	\begin{center}
		\includegraphics[width=0.95\textwidth]{./figures/10statistik.pdf}
	\end{center}
	\caption[Histogramm der Zählstatistik mit Klassen der Größe 10]{
		Diese Graphik
		beinhaltet das normierte Histogramm der Messreihe aus
		\autoref{tab:zahlstatistik}. Hier entspricht $N$ dem Stichprobenumfang und $h$
		der Breite der Klassen des Histogramms. Darüber hinaus wurde eine
		Normalverteilung angelegt unter Verwendung des Mittelwert $\mu$ \& der
		Standardabweichung $\sigma$ vom Messreihensatz. Alle hier in der Box
		angeführten Variablen haben Einheit 1
	}\label{fig:10statistik}
\end{figure}

\subsection{Bestätigung des Abstandsgesetzes}

Um die Daten aus \autoref{tab:abstandsgesetz} mit dem Abstandsgesetz, siehe
\autoref{eq:abstandsgesetz}, vergleichen zu können werden zunächst die
Zählraten $z_i$ gemittelt und dessen Standarderror berechnet. Nun wird die
gemittelte Zählrate $z$ gegen den Abstand der Quelle $l_{\mathrm{Quelle}}$
aufgetragen. Damit das Verhalten der Daten einfacher Analysiert werden kann
wird auch ein Fit des Abstandsgesetz gemacht, wobei $k$ der
Proportionalitätskonstanten des Abstandsgesetz entspricht.

\begin{figure}[H]
	\begin{center}
		\includegraphics[width = 0.95\textwidth]{figures/abstandsgesetz.pdf}
	\end{center}
	\caption[Abstandsgesetz einer \ch{_{38}^{90}Sr} Probe]{
		In dieser Graphik ist das
		Abstandsgesetz bei einer \ch{_{38}^{90}Sr} Probe graphisch dargestellt, durch
		das Auftragen der Zählrate $z$ über dem Quellenabstand $l_{\mathrm{Quelle}}$.
		Dabei wurden die Daten aus \autoref{tab:abstandsgesetz} entnommen. Zudem wurde
		das Abstandsgesetz, siehe \autoref{eq:abstandsgesetz}, mit diesen Datenpunkten
		gefittet (blaue Kurve), dabei ist $k$ die Proportionalitätskonstante.
	}\label{fig:abstandsgesetz}
\end{figure}

\subsection{Bestimmung der Endpunktsenergie über Absorption in Aluminium}\label{sec:Auswertung_alu}

Um mit \autoref{eq:Endpunktsenergie} die Endpunktsenergien bestimmen zu können
müssen aus den Daten die Absorptionskoeffizienten der verschiedenen
$\beta$-Emittern durch eine Überlagerung mehrerer Exponentialfunktionen, nach
dem Beer-Lambertschen Gesetzt, siehe \autoref{eq:beerschesgesetzt}, durch einen
Fit bestimmt werden.

Da jedoch diese Gleichung exponentiell von der Eingangsgröße der Dicke $z$
abhängt bietet es sich an, bei dieser Gleichung den Logarithmus zu verwenden
und diese Gleichungen zu linearisieren. Dieser Teil der Auswertung wurde
bereits während des Experiments vom Betreuer durchgeführt. Die eigene
Auswertung lieferte keine sich mit den Daten deckende Fitfunktion, weshalb nun
die Resultate vom Betreuer genommen werden.

\begin{figure}[H]
	\begin{center}
		\includegraphics[width = 0.95\textwidth]{figures/aluminiumabsorbtion.png}
	\end{center}
	\caption[Absorptionskurve von \ch{Al} bei $\beta$-Strahlung]{
		Diese Kurve beinhaltet die
		Absorptionskurve von \ch{Al} bei $\beta$-Strahlung in einfach logarithmischer
		Darstellung. In diesem Diagramm ist $z$ die Dicke der Aluminiumplatte und
		\emph{Intensity} ist die Zählrate der durchdrungenen Teilchen. Dies wurde durch
		eine Superposition von 3 Exponentialfunktionen vom Betreuer gefittet.
	}\label{fig:alu_absorption}
\end{figure}

\begin{figure}[H]
	\begin{center}
		\includegraphics[width = 0.95\textwidth]{figures/aluminiumdoppellog.png}
	\end{center}
	\caption[Doppelt-logarithmische Darstellung der Absorptionskurve von \ch{Al} bei
		$\beta$-Strahlung]{
		Diese Kurve ist doppelt-logarithmische Darstellung der
		Absorptionskurve von \ch{Al} bei $\beta$-Strahlung. In diesem Diagramm ist $z$
		die Dicke der Aluminiumplatte und \emph{Intensity} ist die Zählrate der
		durchdrungenen Teilchen. Dies wurde durch eine Superposition von 2
		Exponentialfunktionen vom Betreuer gefittet.
	}\label{fig:alu_doppellog}
\end{figure}

Hier werden die durch eine Superposition von 2 Exponentialfunktionen gefitteten
Absorbtionskurven angeführt.

\begin{figure}[H]
	\begin{center}
		\includegraphics[width=0.95\textwidth]{figures/absorption2er.pdf}
	\end{center}
	\caption[Absorptionskurve von \ch{Al} bei $\beta$-Strahlung (2er Superposition
		Experimentatoren)]{
		Diese Kurve beinhaltet die Absorptionskurve von \ch{Al} bei
		$\beta$-Strahlung in einfach logarithmischer Darstellung. In diesem Diagramm
		ist $z$ die Dicke der Aluminiumplatte und \emph{Intensity} ist die Zählrate der
		durchdrungenen Teilchen. Dies wurde durch eine Superposition von 2
		Exponentialfunktionen von den Experimentatoren gefittet.
	}\label{fig:2er_alu_abs}
\end{figure}

\begin{figure}[H]
	\begin{center}
		\includegraphics[width=0.95\textwidth]{figures/absorption2log2er.pdf}
	\end{center}
	\caption[Doppelt-logarithmische Darstellung der Absorptionskurve von \ch{Al} bei
		$\beta$-Strahlung (2er Superposition Experimentatoren)]{
		Diese Kurve ist
		doppelt-logarithmische Darstellung der Absorptionskurve von \ch{Al} bei
		$\beta$-Strahlung. In diesem Diagramm ist $z$ die Dicke der Aluminiumplatte und
		\emph{Intensity} ist die Zählrate der durchdrungenen Teilchen. Dies wurde durch
		eine Superposition von 2 Exponentialfunktionen von den Experimentatoren
		gefittet.
	}\label{fig:2er_alu_abs_doppel}
\end{figure}

Hier werden die durch eine Superposition von 3 Exponentialfunktionen gefitteten
Absorbtionskurven angeführt.

\begin{figure}[H]
	\begin{center}
		\includegraphics[width=0.95\textwidth]{figures/absorption3er.pdf}
	\end{center}
	\caption[Absorptionskurve von \ch{Al} bei $\beta$-Strahlung (3er Superposition
		Experimentatoren)]{
		Diese Kurve beinhaltet die Absorptionskurve von \ch{Al} bei
		$\beta$-Strahlung in einfach logarithmischer Darstellung. In diesem Diagramm
		ist $z$ die Dicke der Aluminiumplatte und \emph{Intensity} ist die Zählrate der
		durchdrungenen Teilchen. Dies wurde durch eine Superposition von 3
		Exponentialfunktionen von den Experimentatoren gefittet.
	}\label{fig:3er_alu_abs}
\end{figure}

\begin{figure}[H]
	\begin{center}
		\includegraphics[width=0.95\textwidth]{figures/absorption2log3er.pdf}
	\end{center}
	\caption[Doppelt-logarithmische Darstellung der Absorptionskurve von \ch{Al} bei
		$\beta$-Strahlung (3er Superposition Experimentatoren)]{
		Diese Kurve ist
		doppelt-logarithmische Darstellung der Absorptionskurve von \ch{Al} bei
		$\beta$-Strahlung. In diesem Diagramm ist $z$ die Dicke der Aluminiumplatte und
		\emph{Intensity} ist die Zählrate der durchdrungenen Teilchen. Dies wurde durch
		eine Superposition von 3 Exponentialfunktionen von den Experimentatoren
		gefittet.
	}\label{fig:3er_alu_abs_doppel}
\end{figure}
Die nun durch den Fit ermittelten Absorptionskoeffizienten werden nun hier
angeführt:

\begin{enumerate}
	\item $\mu_1 = $ \SI{832.2498}{\per\cm}
	\item $\mu_2 = $ \SI{76.0487}{\per\cm}
	\item $\mu_3 = $ \SI{15.3962}{\per\cm}
\end{enumerate}

Mit diesen Werten lassen sich mit \autoref{eq:Endpunktsenergie} die
Endpunktsenergien berechnen:

\begin{enumerate}
	\item $E_1 = $ \SI{0.1273}{\mega\electronvolt}
	\item $E_2 = $ \SI{0.7128}{\mega\electronvolt}
	\item $E_3 = $ \SI{2.2512}{\mega\electronvolt}
\end{enumerate}

Da diese Werte von der Software ohne Unsicherheit ausgegeben wurden, muss diese
als implizit angenommen werden.

\subsection{Aufnahme des Energiespektrums von \texorpdfstring{$\beta$}{beta}
	Strahlung mit Magnetspektrometer}

Um eine Energiespektrum darstellen zu können müssen die Werte aus
\autoref{tab:magnetspektrometer} mittels \autoref{eq:lorentzimpuls} zu Impulsen
$p$ oder mittels \autoref{eq:energieimpulsrelation} zur Energien $E$
transformiert werden. Die erhaltenen Werte für $p$ und $E$ sind in
\autoref{tab:magneto_E_p} zu finden.

\begin{table}[H]
	\caption[Energie- und Impulswerte der $\beta$-Strahlung einer \ch{^{22}_{11}Na} Probe]{
		Errechneten Energien $E$ und Impulse $p$ der $\beta$-Strahlung einer
		\ch{^{22}_{11}Na} Probe vom Magnetspektrometer, mit Daten aus
		\autoref{tab:magnetspektrometer} und der Anwendung der
		\hyperref[eq:energieimpulsrelation]{Energieimpulsbeziehung} sowie
		\hyperref[eq:lorentzimpuls]{Lorentzkraft}                                                          \\
		$E \dots$ Energie der $\beta$-Strahlung einer \ch{^{22}_{11}Na} Probe \\
		$p \dots$ Impuls der $\beta$-Strahlung einer \ch{^{22}_{11}Na} Probe  \\
	}\label{tab:magneto_E_p}
	\centering
	\begin{tblr}{colspec={S[table-format=1.3(2)]S[table-format=1.3(2)]}}
{{{$E_{\mathrm{kin}}$ / \si{\mega\electronvolt}}}} & {{{$p$ / \si{\mega\electronvolt}}}}\\
0.004(1) & 0.067(8)\\
0.022(4) & 0.151(13)\\
0.047(7) & 0.223(17)\\
0.083(11) & 0.30(3)\\
0.122(15) & 0.37(3)\\
0.17(2) & 0.45(3)\\
0.22(3) & 0.52(4)\\
0.28(3) & 0.60(4)\\
0.34(4) & 0.68(5)\\
0.40(4) & 0.75(5)\\
0.46(5) & 0.82(6)\\
0.52(5) & 0.90(6)\\
0.59(6) & 0.97(7)\\
\end{tblr}

\end{table}

Mit den Werten aus \autoref{tab:magneto_E_p} und
\autoref{tab:magnetspektrometer} lässt sich durch Auftragen der Anzahl an
Zerfällen $n$ über den Impulsen $p$ und Energien $E$ ein Energiespektrum
erstellen. Um den Peak numerisch zu bestimmen, wurden die Daten mit einer
Gaussverteilung, welche einen zusätzlichen Amplitudenparameter $A$ hat,
gefittet, wie in folgender \autoref{fig:magneto_E_p} ersichtlich.

\begin{figure}[H]
	\begin{center}
		\includegraphics[width = 0.95\textwidth]{figures/energiespektrum.pdf}
	\end{center}
	\caption[Energie- und Impulsspektogram der $\beta$-Strahlung einer \ch{^{22}_{11}Na}
		Probe]{
		Erhaltenes Energiespektrum (violette) und Impulsspektrum (braun) der
		$\beta$-Strahlung einer \ch{^{22}_{11}Na} Probe. Mithilfe der Energien $E$, den
		Impulsen $p$, aus \autoref{tab:magneto_E_p}, und der Anzahl der Zerfälle $n$,
		aus \autoref{tab:magnetspektrometer}, wurden Spektren erstellt und mit einer
		Gaussverteilung, welche einen zusätzlichen Amplitudenparameter $A$ hat,
		gefittet
	}\label{fig:magneto_E_p}
\end{figure}

\subsection{Aufnahme des komplexen \texorpdfstring{$\gamma$}{gamma} Spektrums
	und seinen Zerfallsprodukten}

Nach der Kalibrierung in \autoref{sec:aufname_Ra_zerfallsreihe} und der
Aufnahme des \(\gamma\)-Spektrum der \ch{^{226}_{88}Ra} Zerfallsreihe
\autoref{sec:aufbau_Magnetfeldspektrometer}, wurden die erhaltenen Daten in
\autoref{fig:Ra226zerfallsreihe} dargestellt. Zudem wurde die Peaks per Hand
ausgewertet und in der Graphik eingezeichnet, daher wurden die Unsicherheiten
großzügig angenommen. Für diese Operationen wurde die ``\emph{Leybold Cassy-Lab
	2}`` Software genutzt.

\begin{figure}[H]
	\begin{center}
		\includegraphics[width = 0.95\textwidth]{figures/Ra226kennlinien.png}
	\end{center}
	\caption[Energiespektrum der $\gamma$-Strahlung einer \ch{^{226}_{88}Ra} Probe]{
		Aufgezeichnetes Gammaspektrum der \ch{^{226}_{88}Ra} Quelle und dessen
		Zerfallsprodukten, mit ausgemessen und markierten Energiepeaks.
	}\label{fig:Ra226zerfallsreihe}
\end{figure}

Die eingezeichneten Spitzenwerte werden nun übersichtlich in
\autoref{tab:raw_peaks} nochmals angeführt.

\begin{table}[H]
	\caption[Erhaltene Peaks bei \ch{_{88}^{226}Ra} Energiespektrum] {
		Erhaltene Peaks bei
		\ch{_{88}^{226}Ra} Energiespektrum \\
		$E \dots$ ist die Energie der Peaks im Gammaspektrum einer \ch{^{226}_{88}Ra} Probe
		mit einer Unsicherheit von \SI{5}{\kilo\electronvolt}
	}\label{tab:raw_peaks}
	\centering
	\begin{tblr}{colspec = {S[table-format = 3.1]}}
		{{{{ $E$ / \si{\kilo\electronvolt}}}}} \\
		20                                     \\
		50                                     \\
		83                                     \\
		191                                    \\
		248                                    \\
		302                                    \\
		360                                    \\
		609                                    \\
	\end{tblr}
\end{table}

\section{Diskussion}\label{sec:diskussion}

Die qualitative Analyse der Zählraten bei unterschiedlichen
Abschirmungsmaterialien entspricht den Erwartungen. Es wird klar ersichtlich,
dass eine dickere Probe des selben Materials die Abschirmung erhöht. Auch wird
deutlich, dass Aluminium, als dichteres Medium eine bessere Abschirmung bietet,
als eine Probe mit vergleichbarer Dicke aus Plastik.

Laut Bedienungsanleitung beträgt die Einsatzspannung des Zählrohrs
\SIrange{350}{380}{\volt} und die maximale Betriebsspannung
\SI{600}{\volt}~\cite{zaehlrohrdoku}. Der Vergleich mit
\autoref{fig:zaelrohrchar} zeigt, dass das erhaltene Spektrum in diesem Bereich
annähernd konstant ist.

Die Betrachtung der erhaltenen Histogramme der Zählstatistik zeigt, dass diese,
wie erwartet, normalverteilt sind.

\autoref{fig:abstandsgesetz} zeigt, dass die erhaltenen Messwerte für das Abstandsgesetz nur leicht über dem gefitteten
quadratischenVerlauf liegen.

Beim betrachten der \autoref{fig:alu_absorption} wird klar ersichtlich, dass
eine dickere Abschirmung zu einem exponentiellen Abfall der verzeichneten
Zählrate hervorbringt, was laut dem Lambert-beersches Gesetz, siehe
\autoref{eq:beerschesgesetzt}, zu erwarten war. Die Lösungsstrategien können zu
äußerst verschiedenen Resultaten führen. Es kommt, dabei darauf an wie viele
Exponentialfunktionen superponiert werden und welche Anfangswerte und Bounds
man beim Fitten nimmt. Deshalb lässt sich nur um eine rein qualitative Aussage
handelt. Im konkreten Fall wurden 2 Fitparameter verwendet, wie bereits in
\autoref{sec:Auswertung_alu} erwähnt, da sich dies nach verschiedenen
Auswertungsversuchen und einem kurzen Gespräch mit dem Betreuer für das
verwendete System am effizientesten erwies. Dennoch sind die Fit für drei
Parameter vom Betreuer und der Experimentatoren angeführt um dem Leser die
Gegenüberstellung machen lassen zu können.

%todo Max ließ bitte nochamol schnell übern neuen Absatz oben drüber obs so passt

Der theoretische Wert für den Peak des $\beta$-Spektrums beträgt
\SI{511}{\kilo\electronvolt} \cite[]{leifi}. Der erhaltene Wert des
Impulsspektrums, von \SI{567(6)}{\kilo\electronvolt}, liegt also in der
richtigen Größenordnung. Die Verschiebung des Peaks kann dadurch erklärt
werden, dass die Hallsonde nur auf die entsprechende Halterung gelegt wurde und
nicht besser befestigt wurde. Dadurch hatte diese einen leichten Spielraum und
konnte so während des Versuchs durch eine leichte Erschütterung verschoben
werden. Dies ist an der verzeichneten Erhöhung des Mittelwerts ersichtlich.

Ein Vergleich in nachfolgender \autoref{tab:vergleich} der erhaltenen Werte des
$\gamma$ - Spektrums mit den entsprechenden Literaturwerten zeigt, dass die
meisten Literaturwerte im Fehlerintervall enthalten sind. Es ist auch sichtbar,
dass die größeren Abweichungen besonders im mittleren Bereich verzeichnet
werden, was durch die Kalibrierung erklärbar ist.

\begin{table}[H]
	\caption[Vergleich der erhaltenen Peaks beim \ch{_{88}^{226}Ra} mit den entsprechenden
		Literaturwerten]{
		Vergleich der erhaltenen Peaks beim \ch{_{88}^{226}Ra} mit
		den entsprechenden Literaturwerten~\cite{Radium}
		\\
		$E \dots$ sind die Energien der gemessen Peaks im Gammaspektrum einer \ch{^{226}_{88}Ra} Probe
		mit einer Unsicherheit von \SI{5}{\kilo\electronvolt} \\
		$E_{\mathrm{lit}} \dots$ sind die Literaturwerte der Energien der Peaks im Gammaspektrum einer \ch{^{226}_{88}Ra} Probe
	}
	\centering
	\begin{tblr}{colspec = {S[table-format = 3.1]QS}}
		{{{\(E\) / \si{\kilo\electronvolt}}}} & Substanz                              & {{{\(E_{\mathrm{lit}}\) / \si{\kilo\electronvolt}}}} \\
		20                                    & \ch{^{212}_{82}Pb}                    & 19.6                                                 \\
		50                                    & \ch{^{214}_{83}Bi}                    & 53                                                   \\
		83                                    & \ch{^{214}_{84}Po}|\ch{^{214}_{83}Bi} & \numlist{79.3;77.1}                                  \\
		191                                   & \ch{^{222}_{86}Rn}                    & 186                                                  \\
		248                                   & \ch{^{214}_{83}Bi}                    & 242                                                  \\
		302                                   & \ch{^{214}_{83}Bi}                    & 295                                                  \\
		360                                   & \ch{^{214}_{83}Bi}                    & 352                                                  \\
		609                                   & \ch{^{214}_{84}Po}                    & 609                                                  \\
	\end{tblr}
	\label{tab:vergleich}
\end{table}

Um die Genauigkeit der erhaltenen Ergebnisse weiter zu verbesser könnte bei
allen Auswertungen die Totzeit des Zählrohr berücksichtigt werden.

\newpage

\section{Zusammenfassung}

Hier werden nochmals alle Ergebnisse dieser Experimentenfolge aufgelistet:

\begin{table}[H]
	\caption[Erhaltene Zählraten bei verschiedenen Abschirmungsmaterialien]{
		Erhaltene
		Zählraten bei verschiedenen Abschirmungsmaterialien bei einer Torzeit von
		\SI{10}{\second} und einem Abstand der radioaktiven Quelle von \SI{15(2)}{\mm}.
		Zählraten sind exakt.                                                                                                              \\
		\(z_{Luft} \dots\) erhaltene Zählrate ohne Abschirmung                                                                             \\
		\(z_{\mathrm{Papier}} \dots\) erhaltene Zählrate mit einem Blatt Papier als Abschirmung                                    \\
		\(z_{\mathrm{Lineal}} \dots\) erhaltene Zählrate mit einem Lineal als Abschirmung, Dicke = \SI{2.1(0.05)}{\mm}             \\
		\(z_{\mathrm{CD}} \dots\) erhaltene Zählrate mit einer CD und zugehörigen Soulcase als Abschirmung                     \\
		\(z_{\mathrm{Alu \num{0.4}}} \dots\) erhaltene Zählrate mit mit einem Aluminiumblech als Abschirmung, Dicke = \SI{0.4(0.05)}{\mm} \\
		\(z_{\mathrm{Alu \num{0.8}}} \dots\) erhaltene Zählrate mit mit einem Aluminiumblech als Abschirmung, Dicke = \SI{0.8(0.05)}{\mm} \\
		\(z_{\mathrm{Alu \num{4}}} \dots\) erhaltene Zählrate mit mit einem Aluminiumblech als Abschirmung, Dicke = \SI{4.0(0.05)}{\mm} \\
	}
	\begin{center}
		\begin{tblr}{colspec={S[table-format=3.1]S[table-format=3.1]S[table-format=2.1]S[table-format=1.1]S[table-format=2.1]S[table-format=2.1]S[table-format=1.1]}}
{{{$z_{\mathrm{Luft}}$ / \si{\cps}}}} & {{{$z_{\mathrm{Papier}}$ / \si{\cps}}}} & {{{$z_{\mathrm{CD}}$ / \si{\cps}}}} & {{{$z_{\mathrm{Lineal}}$ / \si{\cps}}}} & {{{$z_{\mathrm{Alu \num{0.4}}}$ / \si{\cps}}}} & {{{$z_{\mathrm{Alu \num{0.8}}}$ / \si{\cps}}}} & {{{$z_{\mathrm{Alu \num{4}}}$ / \si{\cps}}}}\\
241.6 & 167.3 & 19.4 & 9.6 & 55.1 & 15.5 & 2.3\\
250.3 & 158.7 & 21.7 & 9.8 & 56.6 & 16.3 & 2.7\\
253.0 & 148.6 & 21.4 & 9.4 & 52.9 & 14.4 & 2.9\\
248.5 & 166.5 & 22.8 & 9.6 & 61.7 & 14.5 & 2.5\\
248.3 & 164.3 & 21.3 & 9.5 & 54.2 & 15.4 & 2.4\\
\end{tblr}

	\end{center}
\end{table}

Die Zählrohrcharakteristik ist in \autoref{fig:zaelrohrchar} ersichtlich.

Gemessene Absorptionskoeffizienten von \ch{Al}:
\begin{enumerate}
	\item $\mu_1 = $ \SI{832.2498}{\per\cm}
	\item $\mu_2 = $ \SI{76.0487}{\per\cm}
	\item $\mu_3 = $ \SI{15.3962}{\per\cm}
\end{enumerate}

Gemessene Endpunktsenergien von \ch{Al}:
\begin{enumerate}
	\item $E_1 = $ \SI{0.1273}{\mega\electronvolt}
	\item $E_2 = $ \SI{0.7128}{\mega\electronvolt}
	\item $E_3 = $ \SI{2.2512}{\mega\electronvolt}
\end{enumerate}

Der wahrscheinlichste Wert des Impulsspektrums einer \ch{^{22}_{11}Na} Probe
lag bei \SI{567(6)}{\kilo\electronvolt}.

\begin{table}[H]
	\caption[Vergleich der erhaltenen Peaks beim \ch{_{88}^{226}Ra} mit den entsprechenden
		Literaturwerten]{
		Vergleich der erhaltenen Peaks beim \ch{_{88}^{226}Ra} mit
		den entsprechenden Literaturwerten~\cite{Radium}
		\\
		$E \dots$ sind die Energien der gemessen Peaks im Gammaspektrum einer \ch{^{226}_{88}Ra} Probe
		mit einer Unsicherheit von \SI{5}{\kilo\electronvolt} \\
		$E_{\mathrm{lit}} \dots$ sind die Literaturwerte der Energien der Peaks im Gammaspektrum einer \ch{^{226}_{88}Ra} Probe
	}
	\centering
	\begin{tblr}{colspec = {S[table-format = 3.1]QS}}
		{{{\(E\) / \si{\kilo\electronvolt}}}} & Substanz                              & {{{\(E_{\mathrm{lit}}\) / \si{\kilo\electronvolt}}}} \\
		20                                    & \ch{^{212}_{82}Pb}                    & 19.6                                                 \\
		50                                    & \ch{^{214}_{83}Bi}                    & 53                                                   \\
		83                                    & \ch{^{214}_{84}Po}|\ch{^{214}_{83}Bi} & \numlist{79.3;77.1}                                  \\
		191                                   & \ch{^{222}_{86}Rn}                    & 186                                                  \\
		248                                   & \ch{^{214}_{83}Bi}                    & 242                                                  \\
		302                                   & \ch{^{214}_{83}Bi}                    & 295                                                  \\
		360                                   & \ch{^{214}_{83}Bi}                    & 352                                                  \\
		609                                   & \ch{^{214}_{84}Po}                    & 609                                                  \\
	\end{tblr}
\end{table}

\newpage

\printbibliography
\listoffigures
\listoftables
\end{document}
